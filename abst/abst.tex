\documentclass[twocolumn]{jsarticle}

% 独自のクラスファイルを使うときは,相対パスで指定する必要があります.
% カレントディレクトリはリポジトリ直下です.
% \documentclass[fleqn,11pt]{./abst/hoge}

\usepackage[dvipdfmx]{graphicx}
\usepackage{url}

% 独自のパッケージを使うときは,相対パスで指定する必要があります.
% カレントディレクトリはリポジトリ直下です.
% \usepackage{./abst/fuga}

\begin{document}

\title{\huge{GitHub Actions を用いた \LaTeX ビルドおよび\\PDFファイルの生成と配置}}
\author{\Large Kazuki KANAZAWA}
\date{2020年~2月}
\maketitle

\section{背景}

% これはテスト用のファイルです.

\LaTeX で文章を書くとき,ソースファイルをGitでバージョン管理しながら,添削のためにPDFファイルを出力するという流れがしばしば発生する.
しかし,PDFファイルはコミットする前にコンパイルしなければ正しくバージョン管理されず,ヒューマンエラーが発生しやすい.
つまり,ソースファイルと合わせてバージョン管理していたはずのPDFファイルが,その時点において最新版ではない可能性がある.

GitHubにプッシュした時点でPDFファイルを出力するという試みがなされている\cite{raven38,denkiuo604,takuseno}.
これらはGitHub Actions\cite{github-actions}を用いて実現しており,自動化することによってヒューマンエラーの可能性を排除している.
しかし,いずれも1つのPDFファイルを出力することを想定しており,複数の出力が必要な場合には触れられていない.

\section{目的}

以上の背景のもと,ここでは,\LaTeX で文章を書く際に複数のPDFファイルを出力する必要がある状況を想定し,PDFファイルの出力をGitHub Actionsを用いて実現することを目的とする.
また,出力先は各リポジトリのReleasesとする.
これにより,例えば卒業論文およびその要旨をPDFファイルで出力しなければならない状況においても,自動的に複数のPDFファイルが生成された上で配置されるため,問題が起こりにくいと考えられる.


\begin{thebibliography}{99}
\bibitem{raven38} raven38, ``Github Actionsで卒論CIを作る.,'' \url{http://raven38.hatenablog.com/entry/2019/11/08/000328}, 参照Feb. 2, 2020.
\bibitem{denkiuo604} denkiuo604, ``GitHub Actions で TeX のコンパイルと PDF 化をさせようとしたら意外に苦戦した話,'' \url{https://qiita.com/denkiuo604/items/63b8c34a60a2340727b1}, 参照Feb. 2, 2020.
\bibitem{takuseno} takuseno, ``GitHubだけでLatexをコンパイルしてPDFをリリースにアップロードする,'' \url{https://qiita.com/takuseno/items/2b4c4a129205d03526ad}, 参照Feb. 2, 2020.
\bibitem{github-actions} GitHub, ``Features • GitHub Actions,'' \url{https://github.com/features/actions}, 参照Feb. 2, 2020.
\end{thebibliography}
\end{document}
